\capitulo{1}{Introducción}

El desarrollo de software es un proceso complejo que involucra múltiples factores técnicos y organizativos. En el ámbito técnico, es fundamental garantizar el cumplimiento de los requisitos funcionales y no funcionales, tales como mantenibilidad, escalabilidad, eficiencia y calidad del código. A nivel organizativo, la gestión efectiva de equipos de desarrollo, la planificación de tareas y la optimización del tiempo y recursos son aspectos clave para el éxito de un proyecto. Para abordar esta complejidad, se han desarrollado metodologías y herramientas que facilitan la gestión del proceso de desarrollo, promoviendo la colaboración y la mejora continua.

Las metodologías ágiles, como Scrum, Kanban y eXtreme Programming, han demostrado ser eficaces para gestionar proyectos de software de manera flexible e iterativa. Estas metodologías enfatizan la entrega incremental de valor, la integración continua, la colaboración entre equipos y la adaptación constante a los cambios en los requisitos del proyecto. En este contexto, la gestión de repositorios de código, como GitHub y GitLab, desempeña un papel crucial al proporcionar funcionalidades avanzadas para el control de versiones, la gestión de issues, la revisión de código y la automatización de flujos de integración y despliegue continuo.

Sin embargo, la gran cantidad de datos generados en los repositorios de software dificulta la evaluación de la calidad del desarrollo y la documentación de este, así como la identificación de áreas de mejora. En respuesta a esta necesidad, este TFG propone el desarrollo de una aplicación web que analiza la información de un repositorio de GitHub y, mediante la aplicación de buenas prácticas ágiles y métricas de calidad, ofrece asistencia a los desarrolladores para mejorar su proceso de trabajo.

El sistema evaluará aspectos clave del desarrollo, como la velocidad de trabajo (frecuencia de commits e issues cerradas), la documentación continua, la calidad de la documentación de los commits e issues (uso de descripciones, etiquetas, y demás herramientas de documentación)  y la implementación de pruebas automatizadas. Para ello, se utilizarán métricas cuantitativas que permitan visualizar el rendimiento del equipo y la evolución del proyecto a lo largo del tiempo. Además, el sistema proporcionará recomendaciones basadas en prácticas ágiles establecidas en fuentes como "Subway Map to Agile Practices" de Agile Alliance, tales como integración continua, revisión de código, retrospectivas y gestión visual del trabajo.

Este proyecto busca facilitar la adopción de metodologías ágiles y mejorar la calidad del software y su documentación mediante la automatización del análisis de repositorios. A través de este enfoque, los equipos de desarrollo podrán optimizar sus procesos, mejorar la colaboración y garantizar la entrega de software más robusto y bien documentado.
