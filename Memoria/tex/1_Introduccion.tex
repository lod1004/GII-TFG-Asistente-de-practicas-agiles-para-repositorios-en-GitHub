\capitulo{1}{Introducción}

El desarrollo de software es un proceso complejo que involucra múltiples factores técnicos y organizativos. En el ámbito técnico, es fundamental garantizar el cumplimiento de los requisitos funcionales y no funcionales, tales como mantenibilidad, escalabilidad, eficiencia y calidad del código\cite{isoiec25000}. A nivel organizativo, la gestión efectiva de equipos de desarrollo, la planificación de tareas y la optimización del tiempo y recursos son aspectos clave para el éxito de un proyecto. Para abordar esta complejidad, se han desarrollado metodologías y herramientas que facilitan la gestión del proceso de desarrollo, promoviendo la colaboración y la mejora continua.

Las metodologías ágiles, como Scrum\cite{sutherland2014scrum}, Kanban\cite{roe2017kanban} y eXtreme Programming\cite{beck2004extreme}, han demostrado ser eficaces para gestionar proyectos de software de manera flexible e iterativa. Estas metodologías enfatizan la entrega incremental de valor, la integración continua, la colaboración entre equipos y la adaptación constante a los cambios en los requisitos del proyecto. En este contexto, la gestión de repositorios de código, como GitHub y GitLab, desempeña un papel crucial al proporcionar funcionalidades avanzadas para el control de versiones, la gestión de Issues, la revisión de código y la automatización de flujos de integración y despliegue continuo.

Sin embargo, la gran cantidad de datos generados en los repositorios de software dificulta la evaluación de la calidad del desarrollo y la documentación de este, así como la identificación de áreas de mejora. 

Diversos estudios han explorado la evaluación y aplicación de buenas prácticas ágiles en entornos de desarrollo de software. Por ejemplo, el trabajo de Chen, Chen y Hsueh (2024)\cite{Chen2024} analiza cómo la integración de metodologías ágiles en la enseñanza del desarrollo de software mejora la organización y gestión de proyectos a través del uso de plataformas como GitHub. Su estudio destaca la importancia de la gestión de Issues y Commits como indicadores clave del rendimiento de los desarrolladores, así como el papel de las herramientas automatizadas en la evaluación del trabajo en equipo y la documentación del código. Además, enfatizan que el aprendizaje basado en proyectos (Project-Based Learning, PBL) y la incorporación de prácticas ágiles permiten mejorar la capacidad de los estudiantes para abordar problemas complejos y adaptarse a entornos cambiantes. Estos hallazgos refuerzan la necesidad de implementar sistemas que analicen y promuevan el uso de buenas prácticas en repositorios de software, alineándose con el enfoque propuesto en este proyecto para mejorar la calidad del desarrollo mediante métricas objetivas y recomendaciones basadas en metodologías ágiles.

Además, investigaciones recientes han demostrado el impacto positivo de GitHub en la enseñanza de metodologías ágiles y en el desarrollo colaborativo de software. Raibulet y Arcelli Fontana (2018) \cite{raibulet2018409} describen una experiencia en la que estudiantes de ingeniería de software participaron en proyectos colaborativos utilizando GitHub como plataforma de desarrollo. En este estudio, los estudiantes trabajaron en equipos aplicando principios ágiles y emplearon herramientas como SonarQube para evaluar la calidad del software. Los resultados evidenciaron que el uso de GitHub no solo mejoró la gestión del código y la colaboración entre los participantes, sino que también facilitó la integración de buenas prácticas de desarrollo en un entorno educativo. Estos hallazgos refuerzan la relevancia y necesidad de proporcionar herramientas que analicen y promuevan el uso de metodologías ágiles en proyectos de software, alineándose con los objetivos de este TFG.

En respuesta a esta necesidad, este TFG propone el desarrollo de una aplicación web que analiza la información de un repositorio de GitHub y, mediante la aplicación de buenas prácticas ágiles y métricas de calidad, ofrece asistencia a los desarrolladores para mejorar su proceso de trabajo.

El recurso bibliográfico Subway Map to Agile Practices, desarrollado por Agile Alliance, recopila un conjunto de prácticas ágiles ampliamente utilizadas en la industria del software para mejorar la eficiencia y la calidad en el desarrollo de proyectos. Agile Alliance es una organización global sin ánimo de lucro dedicada a promover y difundir los valores y principios ágiles, proporcionando recursos, estudios y guías que facilitan la adopción de estas metodologías en diversos entornos de trabajo. Entre las prácticas ágiles destacadas en esta colección se encuentran la integración continua, la revisión de código, las retrospectivas y la gestión visual del trabajo, todas ellas orientadas a optimizar la colaboración y el flujo de desarrollo. Basándose en estas prácticas, el sistema desarrollado en este proyecto evaluará aspectos clave del proceso de desarrollo, como la velocidad de trabajo (frecuencia de Commits e Issues cerradas), la documentación continua, la calidad de la documentación en Commits e Issues (uso de descripciones, etiquetas y herramientas de documentación) y la implementación de pruebas automatizadas. Para ello, se emplearán métricas cuantitativas que permitirán visualizar el rendimiento del equipo y la evolución del proyecto a lo largo del tiempo, ofreciendo recomendaciones que fomenten la mejora continua y la adopción de buenas prácticas ágiles.

Este proyecto busca facilitar la adopción de metodologías ágiles y mejorar la calidad del software y su documentación mediante la automatización del análisis de repositorios. A través de este enfoque, los equipos de desarrollo podrán optimizar sus procesos, mejorar la colaboración y garantizar la entrega de software más robusto y bien documentado.

En el contexto docente de los Trabajos de Fin de Grado (TFG), las recomendaciones generadas por la aplicación pueden clasificarse en diferentes reglas de metodologías ágiles según su aplicabilidad y alcance. Se incluyen:

DevOps - Automated Build: El repositorio incluye ficheros que automaticen el desarrollo (La aplicación analiza si se incluyen workflows de acciones de GitHub).

DevOps - Version Control: El repositorio aprovecha en todo lo posible las herramientas de GitHub para el control de versiones (La app analiza la frecuencia de los Commits y la calidad de estos (incluyen título, descripción y referencias a Issues).

DevOps, Extreme Programming - Continuous integration: El repositorio tiene señales de integración continua activa (La aplicación analiza si los workflows, en caso de que se incluyan, y los Pull Requests se ejecutan con frecuencia).

Scrum - Definition of Done: El repositorio comprueba que se dan por terminadas las tareas o historias de usuario correctamente (La app observa la diferencia entre Issues cerradas y abiertas, y que las cerradas no se vuelvan a reabrir).

Scrum - Backlog Quality: El repositorio incluye numerosas issues y de calidad para formar el backlog del prpyecto (La aplicación cuenta el nº de issues y analiza que estas estén bien documentadas, incluyendo descripción con imágenes, personas asignadas y etiquetas)

Scrum, Extreme Programming - Iterations: El repositorio se está desarrollando mediante iteraciones o sprints (La aplicación analizará si se usan milestones en Issues y Pull Requests, y la frecuencia de Commits, merges y releases).

Extreme Programming - Velocity: El repositorio tiene indicios de medición de velocidad de trabajo (Se anañizará si se usan etiquetas que indiquen story points y el número de Issues cerradas en cada Milestone y cada 15 días).

Extreme programming - Frequent Releases: El repositorio incluye releases y se van creando nuevas cada cierto tiempo.

Extreme programming - Collective Ownership: Todos los miembros del equipo pueden modificar cualquier parte del repositorio en cualquier momento (La aplicación comprueba que todos los autores hagan Commits y Pull Requests, sean asignados a Issues, hagan de revisores a Pull Requests de otros, así como la actividad en el tiempo de estos autores).