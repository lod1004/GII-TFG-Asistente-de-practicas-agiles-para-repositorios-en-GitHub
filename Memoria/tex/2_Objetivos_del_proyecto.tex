\capitulo{2}{Objetivos del proyecto} \label{sec:objetivos}

Este apartado explica de forma precisa y concisa cuales son los objetivos que se persiguen con la realización del proyecto. Se puede distinguir entre los objetivos marcados por los requisitos funcionales del software a construir y los objetivos no funcionales que se plantean a la hora de llevar a la práctica el proyecto.

\section{Objetivos funcionales}

Estos objetivos se centran en las funcionalidades que debe ofrecer al usuario la aplicación \textit{Asistente de prácticas ágiles para repositorios en GitHub}.

\begin{itemize}
    \item \textbf{Análisis de métricas de calidad de proceso de software de los repositorios}: La aplicación web analiza detalladamente todas las métricas de calidad de proceso y características que pueden resultar útiles de un repositorio durante el proceso de desarrollo de software. Esto ofrece al usuario una gran variedad de sugerencias de mejora del proceso en su repositorio y facilitar su trabajo, por ejemplo, sugiriendo decripciones y detalles de las \textit{issues}, o la frecuencia de subida de \textit{commits}.

    \item \textbf{Ajuste temporal del análisis}: El usuario puede ajustar distintos parámetros termporales del análisis para elegir en qué etapas del desarrollo se analizarán los repositorios. Se puede elegir analizar con intervalos temporales tanto absolutos como relativos, por ejemplo, del primer mes de vida de repositorio al quinto, o desde el segundo cuarto de vida al tercero, respectivamente
    
    \item \textbf{Comparación de las métricas de calidad de proceso obtenidas de los repositorios de referencia}: La aplicación es capaz de, dadas todas las medidas de calidad de proceso de los repositorios que el usuario ha elegido como referencia, comparar para ofrecer al usuario. Esto se hace comparando de forma cuantitativa y visual la aplicación de prácticas de agilidad, que indican en qué aspectos falla el repositorio.
    
    \item \textbf{Prácticas ágiles y medidas de repositorios}: La aplicación utiliza las definiciones de prácticas ágiles recogidas en \cite{agileSubwayMap} para presentar los resultados al usuario tras analizar el repositorio. El usuario puede acceder a URLs con contenido explicativo breve sobre el por qué se analizan ciertas métricas de calidad de proceso para determinar si un repositorio aplica la práctica ágil en cuestión.

    \item \textbf{Asistencia a estudiantes}: Se pueden utilizar otros TFGs púplicos realizados en la Universidad de Burgos como casos de estudio, introduciéndolos en la aplicación y que estos sirvan como referencia para los estudiantes a la hora de realizar sus propios TFG. Por ejemplo, para el desarrollo del repositorio del TFG que describe esta memoria, se han utilizado como referencia algunos proyectos de TFG que tratan sobre herramientas similares.
    

\end{itemize}

\section{Objetivos no funcionales}

Estos objetivos incluyen características de calidad relacionadas con la mantenibilidad y aspectos tecnológicos del proyecto.

\begin{itemize}
  \item \textbf{Mantenibilidad, modularidad y modificabilidad }: Se ha optado por Angular por su arquitectura basada en componentes, su flexibilidad y facilidad para la maquetación y el diseño de interfaces, su integración con formularios reactivos y su estructura dividida en módulos reutilizables. Esto, junto al uso del servicio HTTP para enviar y recibir datos al BackEnd asincrónicamente, gestionando respuestas y errores de forma controlada han permitido diseñar una aplicación web fácil de mantener y modificar.
  
  \item \textbf{Usabilidad de la aplicación web}: La aplicación debe ser agradable a la vista y fácil de entender. Se busca una simplicidad que ofrezca al usuario una experiencia fácil y rápida de entender. Para ellos se busca indicar de forma clara y concisa al usuario lo que está pasando, qué es lo que está haciendo, y qué significa lo que está viendo en todo momento. Esto se logra mediante la implementación de formularios reactivos con validación dinámica y lógica condicional para distintos modos de análisis (por intervalos de tiempo, fechas relativas, etc.) entre otras características, como los tabs de navegación, que hagan la aplicación usable y cómoda para al usuario.

  \item \textbf{Reusabilidad}: Aplicación de principios como DRY (Don't Repeat Yourself), SOLID (Especialmente el principio Open-Closed, por ejemplo para facilitar el añadido de métricas de calidad de proceso y reglas nuevas) y separación de responsabilidades, además del uso de TypeScript para mantener tipado estricto y detección temprana de errores.


\end{itemize}