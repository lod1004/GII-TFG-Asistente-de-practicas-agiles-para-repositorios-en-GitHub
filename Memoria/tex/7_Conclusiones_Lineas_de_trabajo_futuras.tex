\capitulo{7}{Conclusiones y Líneas de trabajo futuras}

El desarrollo de este proyecto ha permitido abordar de forma práctica y teórica el análisis automatizado de la calidad y la agilidad en proyectos software, con especial atención a su aplicación en entornos académicos. A lo largo del trabajo se ha construido una herramienta funcional que integra métricas objetivas, principios metodológicos y capacidades analíticas para evaluar la evolución y el cumplimiento de buenas prácticas en repositorios gestionados mediante GitHub.

\section*{Conclusiones generales}

Una de las principales aportaciones del proyecto ha sido la demostración de que es posible realizar un seguimiento detallado y automatizado de la calidad del desarrollo de un proyecto utilizando exclusivamente datos disponibles públicamente en plataformas como GitHub. Este enfoque presenta importantes ventajas en términos de escalabilidad, reproducibilidad y objetividad, y puede aplicarse tanto en contextos profesionales como educativos.

Desde un punto de vista metodológico, el proyecto ha validado el uso de medidas de calidad de proceso centradas en la actividad, colaboración y mantenimiento como indicadores fiables de la salud y madurez de un proyecto. Asimismo, se ha confirmado que el análisis temporal aporta una dimensión adicional imprescindible para comprender la evolución de trabajo de un equipo y la implementación progresiva de metodologías ágiles.

\section*{Conclusiones técnicas}

Desde el plano técnico, se destacan los siguientes logros:

\begin{itemize}
\item Se ha desarrollado una herramienta capaz de recopilar y analizar datos estructurados desde la API de GitHub, procesando la información relevante de \textit{commits}, \textit{issues}, \textit{pull requests}, \textit{releases}, ficheros \textit{workflow} y de la actividad de los miembros del equipo.
\item Se ha implementado un sistema de métricas y reglas evaluativas, basado en trabajos de referencia como el \textit{Agile Subway Map}, permitiendo una evaluación automática de prácticas ágiles y de calidad técnica del repositorio. Gracias a la naturaleza del código que cumple el principio abierto/cerrado es sencillo agregar nuevas reglas en el futuro.
\item Se ha incorporado un módulo de análisis temporal dinámico, que permite filtrar y segmentar los datos en función de intervalos absolutos o relativos, lo que amplía notablemente las posibilidades de evaluación y seguimiento.
\item Se ha demostrado la viabilidad del uso académico de la herramienta, facilitando tanto la autoevaluación por parte de los estudiantes como la evaluación objetiva por parte del profesorado.
\end{itemize}

\section*{Valoración crítica del proyecto}

Aunque el proyecto ha cumplido con los objetivos establecidos inicialmente, se han identificado ciertos aspectos que podrían mejorarse o extenderse en futuros desarrollos:

\begin{itemize}
\item La interfaz de visualización y presentación de resultados podría enriquecerse mediante el uso de \textit{dashboards} interactivos, gráficas dinámicas y alertas visuales que faciliten una interpretación más intuitiva por parte del usuario.
\item Otra posible mejora sería permitir una configuración más flexible y adaptativa, dependiendo del tipo de proyecto (tamaño del equipo, duración, objetivos pedagógicos), que permita elegir qué reglas se evaluarán o incluir nuevas personalizadas.
\item El sistema podría beneficiarse de la integración con otras plataformas complementarias a GitHub, como Jira o Trello, para obtener una visión más completa del proceso de gestión ágil cuando se utilicen otras herramientas en paralelo.
\end{itemize}

\section*{Líneas de trabajo futuras}

A partir de los resultados obtenidos y de la experiencia adquirida en este proyecto, se proponen varias líneas de trabajo futuro que permitirían ampliar el alcance, mejorar la utilidad y profundizar en la capacidad analítica del sistema desarrollado:

\begin{enumerate}
\item Ampliación del conjunto de medidas y reglas: Incorporar nuevas medidas relacionadas con la calidad del código, como el número de revisiones por Pull Request, cobertura de código, o uso de despliegues.
\item Desarrollo de un modelo de puntuación integral: Diseñar un sistema de puntuación global que integre múltiples dimensiones (actividad, colaboración, mantenimiento, calidad) para ofrecer una evaluación unificada del estado del proyecto.
\item Validación empírica con más proyectos académicos y profesionales: Realizar estudios de caso con distintas asignaturas, universidades o incluso equipos reales del sector para validar la robustez y adaptabilidad del sistema en diversos contextos.
\item Incorporación de aprendizaje automático: Aplicar técnicas de \textit{machine learning} para detectar patrones de comportamiento, predecir cuellos de botella o sugerir automáticamente acciones de mejora basadas en proyectos similares.
\end{enumerate}

\section*{Reflexión final}

En definitiva, este proyecto representa un primer paso hacia el uso de herramientas automatizadas para la evaluación del proceso de desarrollo software en repositorios de GitHub, integrando principios de ingeniería de software, analítica de datos y metodologías ágiles. El enfoque propuesto no solo permite mejorar la gestión técnica de proyectos, sino que también fomenta el aprendizaje reflexivo y la mejora continua en contextos formativos. Las posibilidades de extensión son numerosas y abren la puerta a una línea de investigación y desarrollo con gran potencial académico y profesional.