\apendice{Anexo de sostenibilización curricular}

\section{F.1. Introducción}
Durante el desarrollo de este TFG, centrado en la creación de una aplicación web para el análisis automatizado de prácticas ágiles y métricas de calidad de proceso en proyectos alojados en GitHub, he aplicado diversas competencias de sostenibilidad de forma transversal. Este trabajo ha sido una oportunidad para reflexionar sobre cómo el desarrollo software puede contribuir de manera activa al cumplimiento de los Objetivos de Desarrollo Sostenible (ODS), especialmente al \acrfull{ods9}, centrado en la industria, innovación e infraestructura, y al \acrfull{ods12}.

El análisis del desarrollo software, combinado con el uso de las prácticas ágiles permiten fomentar procesos de mejora continua que reduzcan el desperdicio de recursos humanos y computacionales, a la vez que promueven modelos colaborativos, transparentes y sostenibles. La herramienta desarrollada contribuye a una mayor conciencia en torno a la eficiencia y responsabilidad dentro de los equipos de desarrollo software.

\section{F.2. Competencias de Sostenibilidad en el proyecto}

\subsection{F.2.1. Adquisición de conocimiento de competencias de sostenibilidad}
A través de este proyecto, he sido capaz de relacionar el desarrollo tecnológico con los retos globales de sostenibilidad. La aplicación se enfoca en facilitar el análisis reflexivo de proyectos software, lo que permite a los equipos evaluar sus prácticas y alinear sus metodologías con objetivos más amplios como la eficiencia, la justicia laboral (mediante la mejora del clima de equipo), o el acceso equitativo al conocimiento abierto.

\subsection{F.2.2. Sostenibilidad al tomar decisiones de proyecto}
He priorizado el uso eficiente de recursos en la arquitectura del sistema, optando por tecnologías open-source, integraciones ligeras con la API de GitHub para minimizar el consumo energético y computacional. Se evita la persistencia innecesaria de datos, lo que contribuye al uso responsable de almacenamiento y procesamiento en servidores.

\subsection{F.2.3. Fomento de la participación colectiva}
Este trabajo promueve el uso de métricas abiertas y replicables para evaluar proyectos en GitHub, lo que incentiva la participación colaborativa en comunidades de desarrollo software. Al facilitar la evaluación colectiva de buenas prácticas, la herramienta ayuda a fortalecer ecosistemas de software más transparentes, resilientes y responsables.

\subsection{F.2.4. Ética del proyecto}
Durante el desarrollo se ha seguido un enfoque ético que evita la recopilación innecesaria de datos personales, se promueve el análisis de repositorios públicos y se fomenta el uso responsable de la tecnología como instrumento para mejorar la transparencia y la calidad en el trabajo en equipo, sin fomentar modelos competitivos insostenibles o presiones laborales indebidas.

\subsection{F.2.5. Conciencia sobre Sostenibilidad en el Desarrollo Software}
Una parte clave del proyecto ha sido su valor educativo. La herramienta actúa como recurso de concienciación para estudiantes y desarrolladores que buscan mejorar sus prácticas en el desarrollo software y planificación para el mismo. Fomenta la reflexión sobre el ciclo de vida de los proyectos de software, sus tareas y componentes, la calidad de sus procesos y su sostenibilidad a largo plazo.

\section{F.3. Conclusión}
El desarrollo de este proyecto ha fortalecido mi comprensión de cómo el software no es únicamente una herramienta técnica, sino un elemento transformador con capacidad de influir en dimensiones sociales y económicas. Aplicar competencias de sostenibilidad en el diseño, implementación y propósito de esta herramienta me ha hecho más consciente de las decisiones éticas y técnicas que tomamos como desarrolladores software. Estoy convencido de que este tipo de soluciones tecnológicas son fundamentales para alcanzar un desarrollo sostenible.