\documentclass[a4paper,12pt]{article}
\usepackage[spanish]{babel}
\usepackage[utf8]{inputenc}
\usepackage{graphicx}

% Documento
\begin{document}

% Portada
\begin{titlepage}
    \centering
    \vspace*{4cm}
    {\Huge \textbf{TFG del Grado en Ingeniería Informática}}

    \vspace{1cm}
    {\LARGE Asistente de buenas prácticas para proyectos en GitHub}

    \vspace{2cm}
    \textbf{Autor:} Lucas Olmedo Díez\\
    \textbf{Tutor:} Carlos López Nozal
    \vfill
    \textbf{Fecha:} Junio 2025

    \vspace{3cm}
    \includegraphics[width=0.3\textwidth]{escudoUniversidad.png}

    \vspace{1cm}
    {\LARGE Universidad de Burgos}


\end{titlepage}

% Resumen
\newpage
\renewcommand*\abstractname{Resumen}
\begin{abstract}
    Este proyecto para Trabajo de Fin de Grado desarrolla una aplicación web escrita en Python y Angular destinada a analizar la calidad y eficiencia de desarrollo de repositorios de GitHub mediante la recopilación y evaluación de métricas clave relacionadas con el proceso de desarrollo, documentación y la interacción de los colaboradores. Estas métricas abarcan información cuantitativa como el número de commits, issues, pull requests, releases y el éxito de los workflows de integración continua. Adicionalmente recopilará y analizará información como como la velocidad de desarrollo, la calidad de la documentación de commits y el ciclo de vida de las issues. \newline

    Para ello, la aplicación se conecta a la API de GitHub para extraer información, tomando como entrada el repositorio destinado a analizar y asistir, y uno o varios repositorios adicionales para utilizar como comparación y fuente de objetivos a alcanzar para el repositorio original. \newline
    
    El objetivo principal de este TFG es proporcionar asistencia para aplicar buenas prácticas de documentación continua, metodologías ágiles y flujos de integración y despliegue continuo, facilitando así la mejora de la calidad y el rendimiento de los proyectos analizados. \newline
    
    \textbf{Descriptores:} Métricas de calidad, proceso de desarrollo de software, metodologías ágiles, integración continua, documentación continua, GitHub, análisis de repositorios, aplicaciones web.
\end{abstract}

% Abstract
\newpage
\renewcommand*\abstractname{Abstract}
\begin{abstract}
    This project for the TFG develops an application written in Python and Angular destined to analyze the quality and the developing efficiency of GitHub repositories by collecting and evaluating key metrics related to the developing process, documentation and the colaborator's interactions with the project. These metrics include quantitative information such as the number of commits, issues, pull requests, releases, and the success of the continuous integration workflows. In addition, the application will extract and analyze information like developing speed, documentation quality and the issues' life cycle.\newline

    To do the previously mentioned, the application connects itself to the GitHub Api to extract the information, taking as input the repository to analyze and assist, and one or more repositories to use as an achievements and comparison source\newline
    
    The main objective of this TFG is to give assistance to apply good practices of continuous documentation, agile methodology and continuous integration and deployment flows, making maintaining the project's quality and efficiency easier and more achievable\newline
    
    \textbf{Keywords:} Evolution metrics, software developing process, agile methodology, continuous integration and documentation, GitHub, repository analysis, web applications.
\end{abstract}

% Introducción
\newpage
\section{Introducción}
Placeholder

\end{document}
